\documentclass[../../main]{subfiles}

\begin{document}
\begin{sloppy}
\section{绪论}

    \textbf{数据}: 客观描述事物属性的数, 字符及所有能被输入到计算机中并被程序识别和处理的符号集合

    \textbf{数据元素}: 数据的基本单位, 作为一个整体进行考虑和处理. 根据业务确定具体构成.

    \textbf{数据对象}: 具有相同属性的数据元素的集合. \textbf{不强调数据元素之间的关系}

    \textbf{数据类型}: 1) 原子类型: 不可再分. 2) 结构类型: Struct. 3) 抽象数据类型ADT: 数学化的语言, 定义数据的取值范围, 结构形式, 操作(逻辑结构, 数据运算). 不关心存储结构, 与具体实现无关.

    \textbf{数据结构}: 相互之间存在一种或多种特定\textbf{关系}的\textbf{数据元素}的集合. \textbf{三要素}: 逻辑结构, 存储结构, 数据的运算

\subsection{数据元素三要素}
    
    \textbf{1. 逻辑结构} : 数据元素之间的逻辑关系, 与存储无关. 线性结构(线性表); 非线性结构(集合, 树, 图 \dots). 

    \begin{note}
    哈希表属于存储结构, 有序表(有序线性表)属于逻辑结构
    \end{note}

    1) 集合: 元素之间只有\textbf{同属于一个集合}的关系

    2) 线性结构: \textbf{只存在一对一}关系, 除首末节点, 其余节点只有唯一前驱和唯一后继

    3) 树形结构: \textbf{一对多}关系

    4) 图状结构/网状结构: \textbf{多对多}关系

    \textbf{2. 存储结构}: 也称为物理结构, 是数据结构在计算机中的表示(映像). \textbf{影响}: (a) 存储空间分配的方便程度; (b) 对数据运算的速度. \textbf{主要有}: 顺序存储, 链式存储, 索引存储, 散列存储

    1) 顺序存储: 逻辑上相邻, 物理上也相邻. 优点: 可以随机存取; 缺点: 只能使用一整块存储单元

    2) 链式存储: 逻辑上相邻, 物理上不要求相邻. 使用指针表示元素之间的逻辑关系. 优点: 不会出现碎片现象; 缺点: 只能顺序存取; 指针占据额外存储空间

    3) 索引存储: 建立附加的索引表. 索引表中的每项称为\textbf{索引项}. 一般形式: \textbf{(关键字, 地址)}. 优点: 检索速度快; 缺点: 索引表占据额外空间, 增删数据也要修改索引表, 耗费较多时间

    4) 散列存储: 通过关键字直接计算出存储地址. 优点: 查增删效率高; 缺点: 较差的散列函数会造成冲突, 解决冲突会带来时间空间开销

    \textbf{3. 数据的运算}: 运算的\textbf{定义}针对\textbf{逻辑结构}; \textbf{实现}针对\textbf{存储结构}. 

    \begin{note}
        不同的数据结构, 其逻辑结构, 存储结构不一定不同. EX. 链表实现队列和栈; 图的存储又邻接表, 邻接矩阵
    \end{note}

    \begin{note}
        存储数据时, 要存储数据的值和数据元素之间的关系. 数据的操作方法, 数据元素类型, 存取方法隐含在实现或逻辑结构中, 或可以通过上下文推断, 不需要显式存入.
    \end{note}

\newpage
\subsection{算法}

    \textbf{五个重要特征}: 缺少任何一个都不是算法

    1) 有穷性: 算法要在有穷步完成, 每一步要在有穷时间内完成.

    2) 确定性: 每条指令要有确切含义, 无歧义. 相同输入得到相同输出.

    3) 可行性: 能够通过代码实现.

    4) 输入: 0或多个输入

    5) 输出: \textbf{1或多个}输出

    \textbf{好的算法}: 只要满足上述五个特征, 就算是算法

    1) 正确性: 正确解决问题; 2) 可读性; 3) 健壮性: 对非法数据有反应或处理; 4) 高效率和低存储量需求: 时间空间复杂度低.

    \begin{note}
        程序可以无穷, 只要不关掉.

        确定性的无歧义EX. {48, 76, 76}选择最大的元素 -> 同值时确定最大 -> 顺序前/后最大 (稳定排序)
    \end{note}

    \subsubsection{时间复杂度}
    对于问题规模$n$, 时间复杂度记为:
    \begin{equation*}
        T(n) = O(f(n))    
    \end{equation*}
    \begin{equation*}
        O(1) < O(\log_2^n) < O(n) < O(n \log_2^n) < O(n^2) < O(n^3) < O(2^n) < O(n!) < O(n^n)
    \end{equation*}

    \subsubsection{空间复杂度}
    对于问题规模$n$, \textbf{只考虑与问题规模n相关的额外空间/辅助空间}. 空间复杂度记为:
    \begin{equation*}
        S(n) = O(f(n))
    \end{equation*}
    \begin{note}
        算法\textbf{原地工作}: 所需的辅助空间为$O(1)$
    \end{note}

\newpage
\subsubsection{空间复杂度计算}
    \vspace{-0.7\baselineskip}
    1. $S(n) = O(\mbox{递归深度})$
    \begin{lstlisting}[style = Pseudocode]
    Function fun(n):
        ...
        return fun(n-1)
    \end{lstlisting}

    \vspace{-0.7\baselineskip}
    2. $S(n) = n + (n-1) + \dots + 1 = \frac{n(n+1)}{2} -> O(n^2)$
    \begin{lstlisting}[style = Pseudocode]
    Function fun(n):
        ... 
        int a[n]
        fun(n-1)
        return 
    \end{lstlisting}

\subsubsection{时间复杂度计算}
    \vspace{-0.7\baselineskip}
    \textbf{1. 循环}: 作表

    \begin{lstlisting}[style = Cpp]
    int sum = 0;
    for (int i=1; i<n; i *= 2)
        for (int j=0; j<i; j++)
            sum++;
    \end{lstlisting}

    令基本运算 \verb|sum++|的执行次数为x, 有:

    % new two columns
    \begin{minipage}{0.48\textwidth}
        \begin{tabular}{c|c|c}
            \toprule
                i & j & x \\
            \midrule
                1 & 0 & $2^0$ \\
                $2^1$ & $0 \sim 1$ & $2^1$ \\
                $2^2$ & $0 \sim 2^2 -1$ & $2^2$ \\
                \dots & \dots & \dots \\
                $2^c$ & $0 \sim 2^c -1$ & $2^c$ \\
        \end{tabular}
    \end{minipage}
    \begin{minipage}{0.48\textwidth}
        运行次数之和$\sum x = \frac{1(1 - 2^c )}{1-2} = 2^c - 1$

        $ 2^c < n < 2^{c+1}$

        $ c < \log_2^n < c+1$

        $ \log_2^n - 1 < c < \log_2^n$

        故: $\sum x < n - 1$

        故: $T(n) = O(n)$
    \end{minipage}
    
    \textbf{2. 循环主体中变量参与循环条件判断}

    \begin{lstlisting}[style = Cpp]
    // 1
    int i {1};
    while (i <= n) 
        i *= 2;
    // 2
    int y {5}:
    while ((y+1) * (y+1) < n)
        y++;
    \end{lstlisting}

    1. 令基本运算 \verb|i *= 1|的执行次数为x, 则$2^x \leq n$, 解得$x \leq \log_2^n$, 故$T(n) = O(\log_2^n)$

    2. 令基本运算 \verb|y++|的执行次数为x, 则$y = x+5$, $(x+5+1)^2 < n$, 解得$x < \sqrt[]{n } - 6$, 故$T(n) = O(\sqrt[]{n})$

    \textbf{3. 递归}: 递推求解, 可能需要使用数学归纳
    \vspace{-0.7\baselineskip}
    EX. 
    \begin{equation*}
        T(n) = 1 + T(n-1) = 1 + 1 + T(n-2) = \dots = n-1 + T(1)
    \end{equation*}
    即$T(n) = O(n)$


    
    
    


    
    
\end{sloppy}
\end{document}