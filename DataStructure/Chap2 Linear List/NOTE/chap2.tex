\documentclass[../../../main]{subfiles}

\begin{document}
\begin{sloppy}

\section{线性表}
    
    \textbf{线性表}: 具有\textbf{相同}数据类型的n个\textbf{数据元素}的\textbf{有限序列}. 
    
    n为\textbf{表长}, n=0时, 线性表为\textbf{空表}. 将线性表命名为$L$, 表示为$L = (a_1, a_2, \dots, a_i, a_{i+1}, \dots, a_n)$. i为位序, 1-index, 表示\textbf{第i个元素}; 数组索引为0-index
    
    \textbf{表头元素}: $a_1$, \textbf{表尾元素}: $a_n$. 表头元素表尾元素均唯一. 
    
    除第一个元素, 每个元素有且仅有一个直接前驱; 除最后一个元素, 每个元素有且只有一个直接后继. 

    \textbf{线性表的基本操作: 创销增删查改}

    \begin{lstlisting}[style = Cpp]
    Initial(&L) // 初始化表, 创建一个空的线性表
    Length(L) // 求表长. 返回: 表的长度(数据元素个数)
    LocateElem(L, e) // 按值查找. 返回: L中元素e的位序
    GetElem(L, i) // 按位查找. 返回: L中第i个元素
    ListInsert(&L, i, e) // 插入. 将e插入到L的第i个位置
    ListDelete(&L, i, &e) // 删除. 删除L中第i个元素, 并将删除的元素通过e返回
    PrintList(L) // 打印表
    Empty(L) // 判空. 返回: empty -> true; not empty -> false
    DestroyList(&L) // 销毁. 销毁线性表并释放L所占空间
    \end{lstlisting}

\newpage
\subsection{顺序表}

    \textbf{顺序表}: 线性表的顺序存储. 用一组\textbf{地址连续的单元}依次存储线性表中的数据元素. 逻辑上相邻, 物理上相邻. 
    
    \textbf{特点}: 1) 逻辑顺序与物理顺序相同; 2) 可随机存取; 3) 存储密度大; 4) 拓展容量不方便; 5) 插入删除需要移动大量元素, 时间开销大 
    
    \textbf{优点}: 1) 存储密度大; 2) 可以随机访问: 即通过元素首地址和元素序号可以在O(1)时间内找到指定元素

    \textbf{缺点}: 1) 元素的插入和删除需要移动大量元素, 插入平均需要移动$n/2$个元素, 删除平均需要移动$(n-1)/2$个元素; 2) 顺序存储分配需要一段连续的存储空间

    \textbf{存储结构}

    \begin{lstlisting}[style = Cpp]
    // 静态分配
    typedef struct
    {
        ElemType data[MAXSIZE];
        int maxsize; // 最大长度, 为了方便操作加上, 若保证能访问宏定义MAXSIZE可以省略
        int length; // 当前长度
    }SqList;
    
    // 动态分配
    typedef struct
    {
        ElemType *data;
        int maxsize; // 当前最大长度
        int length; // 当前长度
    }SeqList;
    \end{lstlisting}

    \vspace{-0.5\baselineskip}
    \textbf{初始化表}
    \begin{lstlisting}[style = Cpp]
    // 初始化静态分配顺序表
    void Initial(SqList &L)
    {
        L.maxsize = MAXSIZE;
        L.length = 0;
    }
    
    // 初始化动态分配顺序表
    bool Initial(SeqList &L)
    {
        L.maxsize = MAXSIZE;
        L.length = 0;
        if ((L.data = new Type[MAXSIZE]))
            return true;
        return false;
    }
    \end{lstlisting}

    \newpage
    \textbf{销毁表}
    \begin{lstlisting}[style = Cpp]
    // 销毁动态分配顺序表
    bool DestroyList(SeqList &L)
    {
        L.maxsize = 0;
        L.length = 0;
        delete L.data; // free space
        return true;
    }
    \end{lstlisting}

    \vspace{-0.5\baselineskip}
    \textbf{扩展表}
    \begin{lstlisting}[style = Cpp]
    // 扩展表
    bool IncreaseList(SeqList &L, int n)
    {
        n += L.maxsize; // new data size
        if (Type* new_data = new Type[n]) // try to request space
        {
            for (int i=0; i<L.length; i++) // move elements
                new_data[i] = L.data[i];
            delete L.data; // free space
            L.data = new_data; // change data pointer
            L.maxsize = n; // new data size
            return true;
        }
        else return false;
    }
    \end{lstlisting}

    \vspace{-0.5\baselineskip}
    \textbf{查找}
    \begin{lstlisting}[style = Cpp]
    // 按值查找
    template <typename T>
    int LocateElem(T L, Type e)
    {
        // return order of elemet
        for (int i=0; i<L.length; i++)
            if (L.data[i] == e) return i+1;
        return -1;
    }
    
    // 按位查找
    template <typename T>
    Type GetElem(T L, int i)
    {
        // param i: order of element
        if (i > L.length) // out of range
            return -INT_MAX;
        else return L.data[i-1];
    }
    \end{lstlisting}

    \newpage
    
    \textbf{插入删除}
    \begin{lstlisting}[style = Cpp]
    // 插入元素
    template <typename T>
    bool ListInsert(T &L, int i, Type e)
    {
        // param i: order of element
        // invalid i
        if (i < 1 || i > L.maxsize || L.length == L.maxsize)
            return false;
        for (int j=L.length; j>=i; j--) // move element [i, length] backward
            L.data[j] = L.data[j-1];
        L.data[i-1] = e; // insert element
        L.length++; // update length
        return true;
    }
    
    // 删除元素
    template <typename T>
    bool ListDelete(T &L, int i, Type &e)
    {
        // param i: order of element
        if (i < 1 || i > L.length) // invalid i
            return false;
        e = L.data[i-1]; // save deleting element
        for (int j=i; j<L.length; j++) // move element[i+1, length] forward
            L.data[j-1] = L.data[j];
        L.length--; // update length
        return true;
    }
    \end{lstlisting}

    \vspace{-0.5\baselineskip}
    \textbf{表长\ 打印\ 判空}
    \begin{lstlisting}[style = Cpp]
    template <typename T> // 表长
    int Length(T L)
    {
        return L.length;
    }
    
    template <typename T> // 打印表
    void PrintList(T L)
    {
        for (int i=0; i<L.length; i++)
            cout << L.data[i] << " ";
        cout << endl;
    }
    
    template <typename T> // 判空
    bool Empty(T L)
    {
        return (L.length == 0);
    }
    \end{lstlisting}
        

\newpage
\subsection{线性表的链式表示\ 链表}

\subsubsection{单链表}





\subsubsection{双链表}





\subsubsection{循环单链表}




\subsubsection{循环双链表}





\subsection{静态链表}




\subsection{顺序表和链表的比较}


\end{sloppy}
\end{document}